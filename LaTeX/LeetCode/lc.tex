\documentclass[12pt, a4paper, oneside]{ctexbook}
\usepackage{amsmath, amsthm, amssymb, bm, wallpaper}
\usepackage{graphicx, hyperref, mathrsfs, caption}
\usepackage{float, subfigure, enumerate, ulem, paralist}
\usepackage{cite}

\CTEXsetup[format={\Large\bfseries}]{section}
\linespread{1.5}
\newtheorem{theorem}{定理}[section]
\newtheorem{definition}[theorem]{定义}
\newtheorem{lemma}[theorem]{引理}
\newtheorem{corollary}[theorem]{推论}
\newtheorem{example}[theorem]{例}
\newtheorem{proposition}[theorem]{命题}

\newcommand\zm[2]{\begin{proof}[\textbf{#1}]
    #2
\end{proof}}
\newcommand\jie[2]{\begin{proof}[\textbf{#1}]
    #2
\end{proof}}
\newcommand\bv[1]{\boldsymbol{#1}}
\newcommand\mb[1]{\mathbb{#1}}
\newcommand\mc[1]{\mathcal{#1}}

% \renewcommand{\qedsymbol}{}

\captionsetup{labelformat=default,labelsep=space} %去除冒号

\begin{document}

\title{{\Huge{\textbf{LeetCode中的数学知识}}\\}}
\author{Lollins}
\date{\today}

\maketitle

\pagenumbering{roman}
\setcounter{page}{1}

\begin{center}
    \Huge\textbf{前言}
\end{center}
\par 主要是用于记录刷LeetCode中涉及到的一些数学知识,大部分的内容会参考北大算协的
\href{https://oi-wiki.org/}{OI Wiki}。


\begin{flushright}
    \begin{tabular}{c}
        Lollins \\
        \today
    \end{tabular}
\end{flushright}

\newpage
\pagenumbering{Roman}
\setcounter{page}{1}
\tableofcontents
\newpage
\setcounter{page}{1}
\pagenumbering{arabic}




\end{document}